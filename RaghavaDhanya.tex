
\documentclass[10pt]{article}

\usepackage[utf8]{inputenc}
\usepackage[T1]{fontenc}
\usepackage[a4paper,includeheadfoot, left=.7in,right=.7in,top=.3in,bottom=.2in]{geometry}
\usepackage[compact]{titlesec}
\usepackage{xcolor}
\usepackage[hidelinks]{hyperref}
\usepackage{fontawesome} 
\hypersetup{ colorlinks=false, } 
\titleformat*{\section}{\titlerule[0.8pt]\color{accentblue}\Large\bfseries\scshape}
\pagestyle{empty}
\urlstyle{same} 
\setlength{\parskip}{0pt} 
\setlength{\parsep}{0pt}
\setlength{\headsep}{0pt} 
% \setlength{\topskip}{0pt} 
% \setlength{\topmargin}{0pt} 
\setlength{\topsep}{0pt} 
\setlength{\partopsep}{0pt}
\definecolor{vividred}{RGB}{194,41,65}
\definecolor{anotherred}{HTML}{D32F2F}
\definecolor{accentblue}{HTML}{255CF2}
\begin{document}
\thispagestyle{empty}

\begin{center}
 \textbf{\textsc{\color{anotherred}\Huge Raghava Dhanya}}\\[10pt] %\rule{\textwidth}{.4pt}
\end{center}

%%%%%%%%%%%%%%%%%%%%%
% social %
%%%%%%%%%%%%%%%%%%%%%
\begin{center}
 \href{mailto:raghavadhanya@gmail.com}{\faEnvelope\ raghavadhanya@gmail.com}
 \ | \ %
 \href{tel:9148995472}{\faPhoneSquare\ 9148995472} \ | \ %
 \href{https://github.com/RaghavaDhanya}{ \underline{\faGithubSquare\
 GitHub}} \ | \ \href{https://in.linkedin.com/in/raghavadhanya}{\underline
 {\faLinkedinSquare\ LinkedIn}} \\
 Bengaluru 562157
\end{center}
%%%%%%%%%%%%%%%%%%%%%

%%%%%%%%%%%%%%%%%%%%%
% Education %
%%%%%%%%%%%%%%%%%%%%%
\section{Education} \textbf{Bachelor of Engineering,} computer science
and engineering \hfill [\textit{2014-present}]\\
Aggregate: \textit{75.92\%}\\
Electives: \textbf{\textit{`Pattern Recognition', `Artificial Intelligence',
`Clouds,Grids and Clusters'.}}\\
\textit{Sir M. Visvesvaraya Institute of Technology,} Bengaluru
\medskip
\\
\textbf{Machine Learning} by Stanford University on Coursera.\hfill [\textit
{March 2017 - July 2017}]\\
Certificate earned on July 21, 2017
\medskip\\ 
\textbf{Neural Networks and Deep Learning} by Deeplearning.ai on Coursera.\hfill [\textit
{February 2018 - March 2018}]\\
Certificate earned on March 4, 2018 
%%%%%%%%%%%%%%%%%%%%%

%%%%%%%%%%%%%%%%%%%%%
% skills %
%%%%%%%%%%%%%%%%%%%%%
\section{Skills} Technologies: Android, Machine Learning, Amazon Web
Services, Django, Continuous Integration.\\
Programming languages: C/C++, Java, Python, JavaScript, SQL, Shell,
HTML \& CSS.\\
Others: Numpy, Scipy, Git, Unix/Linux, Windows. 
%%%%%%%%%%%%%%%%%%%%%

%%%%%%%%%%%%%%%%%%%%%
% Experience %
%%%%%%%%%%%%%%%%%%%%%
\section{Experience} \textbf{Technology Developer,} \textit{notNULL},
Bengaluru \hfill [\textit{February 2016 - July 2016}]\\
Worked on developing a verification technology for events.
\medskip
\\
\textbf{Software Development Intern,} \textit{Supertext}, Bengaluru
\hfill [\textit{August 2016 - December 2016}]\\
Developed an android app for company communication with vendors using
parse and firebase as backend. 
%%%%%%%%%%%%%%%%%%%%%

%%%%%%%%%%%%%%%%%%%%%
% Projects %
%%%%%%%%%%%%%%%%%%%%%
\section{Projects}

\textbf{\underline{%\large\href{https://github.com/abhijith0505/Tonite}
{Image Regeneration with Generative Models}}}\hfill[\textit{March 2018 - May 2018}]\\
An approach to use newly introduced CapsNet as a discriminator in Generative Adversarial Network and demonstrate its application using semantic inpainting on MNIST and face images.\\
\textit{Technologies used: Keras, Tensorflow, python.}
\smallskip
\\
\textbf{\underline{\large\href{https://github.com/abhijith0505/Tonite}
{Tonite}}}\hfill[\textit{April 2016}]\\
Wardrobe assistant app to keep track of the clothes and gives purchase
suggestions based on events in user's calendar. Built at Hackerramp
2016 (16th and 17th April ), Myntra office Bangalore.\\
\textit{Technologies used: Android, Java, XML.}
\smallskip
\\
\textbf{\underline{\large\href{https://github.com/RaghavaDhanya/ReMorse}
{ReMorse}}}\hfill[\textit{April 2017}]
\\
A 2D side scrolling game which tries to subconsciously teach Morse Code.\\
\textit{Technologies used: C++, OpenGL, Box2D}
\smallskip
\\
\textbf{\underline{\large\href{https://github.com/RaghavaDhanya/Codesnap}
{Codesnap}}}\hfill[\textit{August 2016}]
\\
A chrome extension which adds copy button to code segments in most
websites, also provides a clipboard history.\\
\textit{Technologies used: Chrome developer tools, JavaScript, CSS,
HTML. }
\smallskip
\\
\textbf{\underline{\large\href{https://github.com/RaghavaDhanya/FileHide}
{FileHide}}}\hfill[\textit{May 2016}]
\\
A cross platform app which hides one file in another, the host will
still work normally.\\
\textit{Technologies used: Python, Tkinter.}
\smallskip
\\
\textbf{\underline{\large\href{https://github.com/RaghavaDhanya/Snake}
{Snake}}}\hfill[\textit{December 2015}]
\\
Simple, classic and cross platform snake game.\\
\textit{Technologies used: Python, Pygame.} 
%%%%%%%%%%%%%%%%%%%%%

%%%%%%%%%%%%%%%%%%%%%%%%%%
% Activities & Interests %
%%%%%%%%%%%%%%%%%%%%%%%%%%
\section{Activities and Interests}
\begin{itemize}
 \itemsep0em
 \item 
 Won First prize in annual departmental project exhibition 2018 for the project "Image Regeneration with Generative Models" out of 40+ teams.
 \item 
 Twice second placed in coding and debugging competitions conducted during college fests.
 \item
 Organized many events as a core member college coding club CodeShack.
 \item
 Regularly take part in Competitive coding, attend Hackathons and Conferences.
 \item
 Volunteered for android app development for college fest Kalanjali-2015.
 \item
 Graphic designing, designed many posters and banners for college events
 \item
 Lance corporal, National Cadet Corps – Air Wing (High school)

\end{itemize}
%%%%%%%%%%%%%%%%%%%%%%%%%%

\end{document}

